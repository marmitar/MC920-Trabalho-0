\section{Implementação}

Toda a parte de processamento de imagem se encontra no código fonte \texttt{lib.py}. Cada subseção mostra no trecho de código qual função nesse arquivo se encontra a implementação da operação. Os trechos apresentados refletem o código no arquivo, mas sem detalhes como cópias de \textit{buffer} e tipagem estática.

Quando possível, a função equivalente do OpenCV também será apresentada, já que as acelerações de GPU podem ser mais facilmente acessadas usando \pyline{cv2.cuda} em vez de apenas \pyline{cv2} \autocite{ref:cvcuda}. Entretanto, as operações foram implementadas apenas com Numpy neste trabalho, visando a familiarização com técnicas de vetorização.

Assuma que as bibliotecas são importadas como:

\begin{minted}{python}
    import numpy as np
    import cv2
\end{minted}

\subsection{Conversão para Monocromático}

A transformação para escala de cinza é feita através da média dos três canais de cores, truncada para inteiro. A matriz resultante é então repetida novamente para os três canais, para que a operação possa ser repetida sem erros de execução do programa.

\begin{figure}[H]
    \centering
    \begin{subfigure}{0.45\textwidth}
        \centering
        \includegraphics[width=6cm]{resultados/colormono.png}
        \caption{\texttt{imagens/color.png}}
    \end{subfigure}%
    \begin{subfigure}{0.45\textwidth}
        \centering
        \includegraphics[width=6cm]{resultados/citymono.png}
        \caption{\texttt{imagens/city.png}}
        \label{fig:res:1}
    \end{subfigure}

    \caption{Imagem em escala de cinza.}
\end{figure}

\begin{listing}[H]
    \begin{minted}{python}
        def grayscale(imagem):
            gray = np.mean(imagem, axis=2).astype(np.uint8)
            return np.stack([gray, gray, gray], axis=2)
    \end{minted}

    \caption{Comando \texttt{monocromatico}}
\end{listing}

Em vez de \pyline{np.mean(imagem, ...)}, a conversão poderia ser implementado também com \pyline{cv2.cvtColor(imagem, cv2.COLOR_BGR2GRAY)} \autocite{ref:cvtcolor}.

\subsection{Negativo da Imagem}

Para essa conversão basta fazer ~\pyline{255 - intensidade}~ para cada píxel. Como 255 também é o maior valor de um \textit{byte}, basta também inverter todos os bits da imagem.

\begin{figure}[H]
    \centering
    \begin{subfigure}{0.45\textwidth}
        \centering
        \includegraphics[width=6cm]{resultados/colorneg.png}
        \caption{\texttt{imagens/color.png}}
    \end{subfigure}%
    \begin{subfigure}{0.45\textwidth}
        \centering
        \includegraphics[width=6cm]{resultados/cityneg.png}
        \caption{\texttt{imagens/city.png}}
    \end{subfigure}

    \caption{Imagem com intensidade invertida.}
\end{figure}

\begin{listing}[H]
    \begin{minted}{python}
        def negativo(imagem):
            return ~imagem
    \end{minted}

    \caption{Comando \texttt{negativo}}
\end{listing}

Em vez de usar o \textit{not} do Numpy, existe também a função \pyline{cv2.bitwise_not(imagem)} \autocite{ref:bitwise_not}.

\subsection{Espelhamento Vertical}

\begin{listing}[h]
    \caption{Comando \texttt{esp.vertical}}

    \begin{minted}{python}
        def espelhamento_vertical(imagem):
            return imagem[::-1]
    \end{minted}
\end{listing}

\begin{figure}[h]
    \centering
    \begin{subfigure}{0.45\textwidth}
        \centering
        \includegraphics[width=6cm]{resultados/colorflip.png}
        \caption{\texttt{imagens/color.png}}
    \end{subfigure}%
    \begin{subfigure}{0.45\textwidth}
        \centering
        \includegraphics[width=6cm]{resultados/cityflip.png}
        \caption{\texttt{imagens/city.png}}
    \end{subfigure}

    \caption{Imagem espelhada verticalmente.}
\end{figure}

Pode ser implementado também com \pyline{cv2.flip(imagem, 0)} \autocite{ref:flip}.

\subsection{Conversão de Intervalo}

Converte o intervalo de intensidade da imagem de [0, 255] para [100, 200].

\begin{listing}[h]
    \caption{Comando \texttt{conv.intervalo}}

    \begin{minted}{python}
        def converter_intervalo(imagem):
            zmin, zmax = np.min(imagem), np.max(imagem)
            img = 100 * (image / (zmax - zmin)) + 100
            return img.astype(np.uint8)
    \end{minted}
\end{listing}

\begin{figure}[h]
    \centering
    \begin{subfigure}{0.45\textwidth}
        \centering
        \includegraphics[width=6cm]{resultados/colorconv.png}
        \caption{\texttt{imagens/color.png}}
    \end{subfigure}%
    \begin{subfigure}{0.45\textwidth}
        \centering
        \includegraphics[width=6cm]{resultados/cityconv.png}
        \caption{\texttt{imagens/city.png}}
    \end{subfigure}

    \caption{Intervalo de intensidade convertido.}
\end{figure}

\subsection{Inversão das Linhas Pares}

Inverte todas as linhas pares horizontalmente.

\begin{listing}[H]

    \begin{minted}{python}
        def inverte_linhas_pares(imagem):
            magem[::2] = image[::2,::-1]
            return imagem
    \end{minted}

    \caption{Comando \texttt{inverte.pares}}
\end{listing}

\begin{figure}[h]
    \centering
    \begin{subfigure}{0.45\textwidth}
        \centering
        \includegraphics[width=6cm]{resultados/colorinvp.png}
        \caption{\texttt{imagens/color.png}}
    \end{subfigure}%
    \begin{subfigure}{0.45\textwidth}
        \centering
        \includegraphics[width=6cm]{resultados/cityinvp.png}
        \caption{\texttt{imagens/city.png}}
    \end{subfigure}

    \caption{Linhas pares invertidas.}
\end{figure}

\subsection{Mosaico} \label{sec:mosaico}

\textcolor{red}{ORDENACAO? PADRAO? IMPL?}

\begin{minted}{bash}
    $ python main.py imagens/baboon.png mosaico padrao.txt
\end{minted}

\begin{figure}[H]
    \centering
    \begin{subfigure}{0.45\textwidth}
        \centering
        \includegraphics[width=6cm]{resultados/colormsc.png}
        \caption{\texttt{imagens/color.png}}
    \end{subfigure}%
    \begin{subfigure}{0.45\textwidth}
        \centering
        \includegraphics[width=6cm]{resultados/baboonmsc.png}
        \caption{\texttt{imagens/babooon.png}}
    \end{subfigure}

    \caption{Mosaico da imagem com \textcolor{red}{ALGUMA COISA}.}
\end{figure}

\begin{listing}[H]
    \caption{Comando \texttt{mosaico ORDENACAO}}

    \begin{minted}{python}
        def mosaico(imagem, ordem):
            ...
    \end{minted}
\end{listing}

