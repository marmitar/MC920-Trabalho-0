\subsection{Negativo da Imagem}

Para essa conversão basta fazer ~\pyline{255 - intensidade}~ para cada píxel. Como 255 também é maior valor de um \textit{byte}, basta também inverter todos os bits da imagem.

\begin{listing}[h]
    \caption{Comando \texttt{negativo}}

    \begin{minted}{python}
        def negativo(imagem):
            return ~imagem
    \end{minted}
\end{listing}

\begin{figure}[h]
    \centering
    \begin{subfigure}{0.45\textwidth}
        \centering
        \includegraphics[width=6cm]{resultados/colorneg.png}
        \caption{\texttt{imagens/color.png}}
    \end{subfigure}%
    \begin{subfigure}{0.45\textwidth}
        \centering
        \includegraphics[width=6cm]{resultados/cityneg.png}
        \caption{\texttt{imagens/city.png}}
    \end{subfigure}

    \caption{Inversão da intensidade.}
\end{figure}

Em vez de usar o \textit{not} do Numpy, existe também \pyline{cv2.bitwise_not(imagem)} \autocite{ref:bitwise_not}.
